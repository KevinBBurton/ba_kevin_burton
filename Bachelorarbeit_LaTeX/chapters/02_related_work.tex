% !TeX root = ../main.tex
% Add the above to each chapter to make compiling the PDF easier in some editors.

\chapter{Related Work}\label{chapter:related_work}

\section{KIA4SM}
Cooperative Integration Architecture for Future Smart Mobility Solutions (KIA4SM) is the project that both laid the groundwork and created the need for these checkpointing and restoring considerations. The thesis by Eckl, Krefft and Baumgarten discusses this topic of advancement of intermodal mobility and the effects of this trend, with the current most common computing infrastructure in mind. They propose a shift from the present heterogeneous architecture to a network of homogeneous ECUs, which is achieved through strong virtualisation on a L4 microkernel-based hypervisor. Therefore, communication and cooperation between said ECUs, even across system boundaries (e.g. with other vehicles or traffic lights), is much easier. This makes smarter mobility in the future possible. Furthermore, hardware consolidation and organic computing is utilised for dynamic task distribution between the virtualised system borders. \cite{kia4sm}

\section{Real-Time Checkpoint Restore}
The first implementation of a checkpoint/restore mechanism embedded in the KIA4SM framework was achieved by Huber in 2016 for the Genode operating system. Because of the real-time criticality of the underlying framework, this component required to accomplish its task in the same time-sensitive manner, hence the name. RTCR was designed to keep watch over at least one target process, checkpoint its address space and thread state, and restore when said process encounters a problem, all without missing schedules. To make this possible, the RTCR needed to be able to access the target's resources, which lead to the design decision of the RTCR always being the parent component to any protected process. The checkpointing mechanism is triggered by observing a change to component state, which is done through establishing a shared memory between RTCR and the target child. Alterations are detected in the following way. The shared dataspace is not backed by physical memory and therefore purely virtual. Whenever the child now tries to write to its designated memory, a page fault is caused, which the RTCR then handles by storing a checkpoint and finally backing the memory with physical pages, resulting in a so-called copy-on-write mechanism.

Restoration of processes is performed by hijacking the native Genode bootstrapping mechanism. Just before regular execution of the main component of the target child, restoration is performed, altering the component objects that were created by the bootstrap and recreating those that were not automatically generated. \cite{rtcr}

\section{Modularisation of the Real-Time Checkpoint Restore Mechanism}
In his 2020 thesis, Fischer iterated on the RTCR concept by refactoring the core mechanism and introducing a modularisation concept, effectively creating a second version of the component that he himself termed RTCRv2. His version features a support for multiple kernels, multi-core capability and modularisation. The latter is designed to ease debugging and further development, by combining all forks on the RTCR repository into one, and introducing the possibility to include or exclude program code by bundling it into a module. 

Furthermore a performance measurement suite is introduced, with which the efficiency of RTCRv2 can be tested in regard of Genode version, underlying microkernel, hardware and most importantly different modules. This is essential for the RTCR, as its performance is the key for its real-time capability. With aforementioned suite, different checkpointing solutions could be evaluated with and without hardware acceleration. \cite{rtcr2}

\section{Website Snapshot Management}
The field in which snapshot management is most prevalent is that of website snapshotting, the most notorious one being the Wayback Machine. The main difference between it and the case of managing checkpoints of processes however, is that for the latter, only the most recent snapshot is relevant. Whereas for websites every single change to the resources that compose them has to be tracked, which provides the user with a snapshot of the site at any given time. For the purpose of designing such a snapshot management system, Chao first lays out all of the factors that affect website state and then proposes that instead of saving all these factors for every snapshot, simply all of the changes to website state are logged. Therefore the snapshot manager is able to roll back the site from the current state. \cite{website}